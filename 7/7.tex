\documentclass{article}
\usepackage{fontspec}
\usepackage{xcolor}
%\usepackage{sagetex}

\usepackage{euler}
\usepackage{amsmath}
\usepackage{unicode-math}


\usepackage[makeroom]{cancel}
\usepackage{ulem}

\setlength\parindent{0em}
\setlength\parskip{0.618em}
\usepackage[a4paper,lmargin=1in,rmargin=1in,tmargin=1in,bmargin=1in]{geometry}

\setmainfont[Mapping=tex-text]{Helvetica Neue LT Std 45 Light}

\newcommand\N{\mathbb{N}}
\newcommand\Z{\mathbb{Z}}
\newcommand\R{\mathbb{R}}
\newcommand\C{\mathbb{C}}
\newcommand\A{\mathbb{A}}

\begin{document}

\begin{center}
  151A---7

  Ricardo J. Acu\~na

  (862079740)
\end{center}\vspace{1.618em}

\paragraph{14} Let $I = [0,1]$ be the closed unit interval. Suppose $f$
is a continuous mapping of $I$ into $I$. Prove that $f(x) = x$ for at
least one $x$ in I.

\uwave{pf.}

Let $g:I\rightarrow I ; x\mapsto (f-1)(x)$. Where $1:I\rightarrow I;
x\mapsto x$.

$f:I\rightarrow I \implies f(x) \geq 0$

$\implies f(0) \geq 0$

$\implies f(1) \leq 1 \implies f(1) - 1 \leq 1 - 1 = 0$

$\implies g(0) = f(0) - 0 \geq 0$

$\implies g(1) = f(1) - 1 \leq 0$

$g$ is continuous as it is the difference of continuous functions.

$[0,1]$ is closed. So, the function $g$ has a zero on $I$ by the
intermediate value theorem.

$\implies \exists x\in I: g(x) = 0$

$\implies f(x)-x = 0$

$\implies f(x) = x$

$\qed$
\end{document}

%%% Local Variables:
%%% mode: latex
%%% TeX-master: t
%%% End:
